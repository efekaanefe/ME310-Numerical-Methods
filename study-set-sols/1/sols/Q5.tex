
\section*{Question 5}

To evaluate \( e^{-10} \), we can use two different equations and compare their accuracy. The equations are:

\begin{enumerate}
    \centering
    \item Equation 1: \( e^{-10} = \sum_{n=0}^{\infty} \frac{(-10)^n}{n!} \)
    \item Equation 2: \( e^{-10} = \frac{1}{\sum_{n=0}^{\infty} \frac{10^n}{n!}} \)
\end{enumerate}

We'll implement these equations in MATLAB using both double and single precision, and then analyze the results in terms of truncation and round-off errors.

\subsection*{Results and Discussion}

The MATLAB code was executed to evaluate \( e^{-10} \) using both equations with double and single precision. The true errors after adding each term (using 100 terms) are as follows:

\begin{table}[h]
    \centering
    \begin{tabular}{|c|c|c|}
    \hline
    Equation & Precision & True Error \\
    \hline
    1 & Double & 3.284342741348423e-13 \\
    1 & Single & 4.9780836e-05 \\
    2 & Double & 1.355252715606881e-20 \\
    2 & Single & 2.4320487e-12 \\
    \hline
    \end{tabular}
    \caption{True Errors for \( e^{-10} \)}
    \label{tab:true_errors}
\end{table}

From Table \ref{tab:true_errors}, we observe that both equations yield very low true errors when using both precisions. However, second equation provides less true error for both single and double precisions.

This difference in accuracy between the two equations can be attributed to the way the terms are added. Equation 1 directly sums the terms of the Taylor series expansion of \( e^{-10} \) and works with smaller numbers for each iteration compared to Equation 2. As a result, it creates more roundoff error.

In terms of accuracy, Equation 2 is better suited for computing \( e^{-10} \) as it involves fewer arithmetic operations and results in slightly lower true errors. 

